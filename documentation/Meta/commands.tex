% !TeX root = ../Hauptdokument.tex
% !TeX encoding = utf8
% !TeX program = pdflatex

% !TeX TXS-program:bibliography = txs:///biber

% !TeX spellcheck = de_DE

% !TeX TXS-SCRIPT = loadRoot
% //Trigger = ?load-this-file
% var filename = app.getCurrentFileName();
% var rootFilename = editor.document().getMagicComment("root");
%
% app.load(app.getAbsoluteFilePath(rootFilename));
% app.load(app.getAbsoluteFilePath(filename));
% TXS-SCRIPT-END


%---------------------------
%   Standardbefehle für das Deckblatt
%---------------------------
\newcommand{\documenttitle}{Sawyer: Peg in a hole}
\newcommand{\einreichungsdatum}{\today}

\newcommand{\documentname}{Felix Schmidt B.Sc.}
\newcommand{\documentmanr}{70472032}

\newcommand{\secondDocumentname}{}
\newcommand{\secondDocumentmanr}{}

\newcommand{\dozent}{Professor Dr.-Ing. Reinhard Gerndt}
\newcommand{\secondDozent}{}

\newcommand{\documentmodul}{Robotics/Cobotics}
%---------------------------
%   Ostfalia Standort Einstellung
%---------------------------

\newboolean{WF}
\newboolean{SUD}
\newboolean{WOB}
\newboolean{SZ}
\setboolean{WF}{true}
\setboolean{SUD}{false}
\setboolean{WOB}{false}
\setboolean{SZ}{false}

%---------------------------
%   Zitieren und Referenzieren
%---------------------------

\NewDocumentCommand{\vollZitat}{m m}{\enquote{#1}(cf. \cite{#2})}
\NewDocumentCommand{\sinnZitat}{m}{cf. \cite{#1}}
\NewDocumentCommand{\abbRef}{m o o}{Fig. \ref{#1}\IfValueT{#2}{ and \ref{#2}}\IfValueT{#3}{ and \ref{#3}}}
\NewDocumentCommand{\tabRef}{m o o}{Table \ref{#1}\IfValueT{#2}{ and \ref{#2}}\IfValueT{#3}{ and \ref{#3}}}

%---------------------------
%   Bilderverzeichnisse, Unterordner müssen separat angegeben werden.
%---------------------------

%{Bilder/}     Unterorder im Kapitelorder(deprecated)
%{../Bilder/}  Bilderverzeichnis im Root.
\graphicspath{{Bilder/}{../Bilder/}}

%---------------------------
%  Notizen Kategorien(optional)
%  Beispiele für neue todo kategorien, kann weg da es nicht zum template gehört...
%---------------------------

%Requirements
\NewDocumentCommand{\requ}{m}{\todo[color=blue!40,nolist]{#1}}

%Kommentare
\NewDocumentCommand{\note}{m}{\todo[inline,color=green!40]{\textbf{Notiz:} #1}}

%Optionaler Text
\NewDocumentCommand{\optional}{m}{\todo[inline,nolist,color=gray!40]{\textbf{Textschnipsel:} #1}}

%Alter Text
\NewDocumentCommand{\alt}{m}{\todo[inline,nolist,color=gray!40]{\textbf{ALT:} #1}}

%Requirements
\NewDocumentCommand{\pruefer}{m}{\todo[color=blue!40]{Frage: #1}}

\NewDocumentCommand{\aufgabenbox}{m}{
	\begin{tcolorbox}[colframe=blue!40, colback=gray!5, coltitle=black]
		#1
	\end{tcolorbox}
}

\NewDocumentCommand{\hinweisbox}{m}{
	\begin{tcolorbox}[colframe=red!40, colback=gray!5, coltitle=black]
		#1
	\end{tcolorbox}
}

%---------------------------
%   Signatur mit Ort, Datum und Unterschrift
%---------------------------

\newcommand*{\SignatureAndDate}[1]{
	\par\noindent\makebox[52mm]{Wolfenbüttel, den \today}\vspace{-5mm}
	\par\noindent\makebox[52mm]{\hrulefill}     \hfill\makebox[65mm]{\hrulefill}
	\par\noindent\makebox[52mm][l]{Ort, Datum}	\hfill\makebox[62mm][l]{#1}
}

%---------------------------
%   Definition des Kapitelformats
%---------------------------

\renewcommand{\chapterformat}{} % Entfernt "Kapitel X" über dem eigentlichen Kapitelnamen

\NewDocumentCommand{\schritt}{m}{
	\begin{table}[H]
		\centering
		\begin{tabular}{c|c|c}
			1 & 2 & 3 \\
			#1
		\end{tabular}
	\end{table}
}

\newcommand{\blitz}{\textup{\Lightning} }
