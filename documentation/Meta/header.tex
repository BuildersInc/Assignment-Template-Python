% !TeX root = ../Hauptdokument.tex
% !TeX encoding = utf8
% !TeX program = pdflatex

% !TeX TXS-program:bibliography = txs:///biber

% !TeX spellcheck = de_DE

% !TeX TXS-SCRIPT = loadRoot
% //Trigger = ?load-this-file
% var filename = app.getCurrentFileName();
% var rootFilename = editor.document().getMagicComment("root");
%
% app.load(app.getAbsoluteFilePath(rootFilename));
% app.load(app.getAbsoluteFilePath(filename));
% TXS-SCRIPT-END


%==========================
% Genutzte Pakete
%==========================
\usepackage{scrhack}
\usepackage[T1]{fontenc}            % Zeichencodierung
\usepackage[utf8]{inputenc}         % Zeichencodierung
\usepackage{lmodern}                % Schriftbildverbesserung
\usepackage{microtype}
% \usepackage[ngerman]{babel}         % Sprache Deutsch
\usepackage[english]{babel}   % Sprache Englisch
% \usepackage[english,ngerman]{babel}
\usepackage[colorinlistoftodos, textsize=tiny,disable]{todonotes}            % ToDos  [disable] zum Ausschalten
\usepackage{listings}
\usepackage{listingsutf8}
\usepackage{graphicx}               % Grafiken
\usepackage{geometry}               % Seitenränder
\usepackage{chngcntr}               % Nummerierungen ohne Kapitelnummer
\usepackage{setspace}               % Zeilenabstand
\usepackage{scrlayer-scrpage}       % Kopf- und Fußzeilen
\usepackage{xparse}                 % Definieren von Befehlen
\usepackage{xspace}
\usepackage{xstring}
\usepackage{acronym} % Fuer Abkuerzungen
\usepackage{multicol}
\usepackage{ifthen}                 % If Abfragen
\usepackage[titles]{tocloft}        % Inhaltsverzeichnis
% (option titles sorgt für standard LaTeX Methode)
\usepackage{longtable}              % Lange Tabellen
\usepackage{csquotes}               % Für Anführungszeichen
\usepackage{colortbl}
\PassOptionsToPackage{hyphens}{url}
\usepackage[hidelinks]{hyperref}    % Links an Verweisen (hidelinks Links sind nicht umranden)
\usepackage[most]{tcolorbox}        % Für schicke Boxen
\usepackage{makecell}
\usepackage{float}
\usepackage{wasysym}
\usepackage{marvosym}
\usepackage{multirow}


\usepackage{subcaption}

\usepackage{varwidth}

\usepackage[style=iso-numeric,  % Literaturverzeichnis
	backend=bibtex,                 % Erzeuger des Literaturverzeichnis
	natbib=true                     % natbib eingeschaltet
]{biblatex}


\usepackage{amsmath}				% Mathematische Formeln

\usepackage{tikz}					% Graphen Erstellen
\usetikzlibrary{shapes}             % B+-Trees
\usetikzlibrary{arrows.meta, positioning} % Serialisierbarkeitsgraphen
\usepackage{pgfplotstable}			% Erweiterung für Tabellen Import
\usepackage{pgfplots}				% Erweiterung zum Plotten
\pgfplotsset{compat=1.15}

\usepgfplotslibrary{external} 		% Cachen von TIKZ Graphen


\usepackage{eurosym}				% Euro zeichen

\usepackage{enumitem}
%\newlist{bulletdescription}{description}
%\setlist[bulletdescription]{font=$\bullet$\scshape\bfseries}

%==========================
% Einstellungen (standard Konfigurationen)
%==========================

%--------------------------
% Seitenlayout
%--------------------------
\geometry{a4paper}                  % A4 Blattgröße
\geometry{twoside=false}            % Einseitiges Dokument (kein Buch)
\geometry{top=2.5cm}                % oberer Seitenrand
\geometry{bottom=2cm}               % unterer Seitenrand
\geometry{left=3cm}                 % linker Seitenrand
\geometry{right=2cm}                % rechter Seitenrand
\geometry{bindingoffset=0cm}        % Bindekorrektur
\geometry{includehead=true}         % Kopfzeile im Textkörper
\geometry{includefoot=true}         % Fußzeile im Textkörper

%--------------------------
% Schriftgrößen
%--------------------------
\KOMAoption{fontsize}{11pt}                 % Schriftgröße
\KOMAoption{headings}{big}
%\addtokomafont{chapter}{\Large}            % Schriftgröße Kapitel
%\addtokomafont{section}{\large}            % Schriftgröße Section
%\addtokomafont{subsection}{\normalsize}    % Schriftgröße Subsection
%\addtokomafont{subsubsection}{\normalsize} % Schriftgröße Subsubsection
%\addtokomafont{paragraph}{\normalsize}     % Schriftgröße Paragraph
%\addtokomafont{subparagraph}{\normalsize}  % Schriftgröße Suparagraph

% Abstand vor paragraph und subparagraph
\RedeclareSectionCommands[
	beforeskip=-1ex
]{paragraph,subparagraph}


%--------------------------
% Überschriften
%--------------------------
\KOMAoption{chapterprefix}{true}        % Prefix vor Kapiteln
\KOMAoption{appendixprefix}{true}       % Prefix vor Kapiteln im Anhang

%--------------------------
% Zeilenabstände
%--------------------------
\onehalfspacing                        % Eineinhalbfacher Zeilenabstand

%--------------------------
% Inhaltsverzeichnis
%--------------------------
\KOMAoption{toc}{bibliography}  % Literaturverzeichnis im Inhaltsverzeichnis
\KOMAoption{toc}{listof}        % Alle Verzeichnisse mit Eigenschaft totoc im
% Inhaltsverzeichnis anzeigen (Abbildungsverzeichnis,
% Tabellenverzeichnis)
\setuptoc{toc}{totoc}           % Inhaltsverzeichnis bekommt eigenschaft totoc (wird demnach
% auch im Inhaltsverzeichnis angezeigt)

%--------------------------
% Nummerierungen
%--------------------------
\KOMAoption{numbers}{noenddot}      % Keine Punkte nach den Nummerierungen
\setcounter{secnumdepth}{3}         % Nummerierung bis zur subsubsection (3)

%--------------------------
% Captions
%--------------------------
\KOMAoption{captions}{oneline}           % Caption Zentriert
\KOMAoption{captions}{figuresignature}   % Abbildungen als Unterschriften
\KOMAoption{captions}{tablesignature}    % Tabellen als Unterschriften
\addtokomafont{captionlabel}{\bfseries}  % Alle Captionlabels werden Fett

%--------------------------
% Abbildungsverzeichnis und Tabellenverzeichnis
%--------------------------
\KOMAoption{listof}{entryprefix}    % Prefix in Tabellen- und Abbildungsverzeichnis
\setlength{\cfttabindent}{0em}      % Einfückung im Tabellenverzeichnis
\setlength{\cftfigindent}{0em}      % Einrückung im Abbildungsverzeichnis

%--------------------------
% Absatzs
%--------------------------
\KOMAoption{parskip}{half}          % Absätze um halben Zeileanstand vergrößert

%--------------------------
% Kopfzeile und Fußzeile
%--------------------------
\automark[section]{chapter}         % chapter und sections sollen verfügbar sein.
\KOMAoption{autooneside}{false}     % stellt ein das optionen von \automark eingesetzt werden.
\KOMAoption{headsepline}{true}      % Trennlinie nach Kopfzeile
\KOMAoption{footsepline}{true}      % Trennlinie vor Fußzeile
\clearpairofpagestyles              % Kopfzeile und Fußzeile leeren
\ihead{\leftmark}                   % Kopfzeile außen
\chead{}                            % Kopfzeile mitte
\ohead{}                            % Kopfzeile innen
\ifoot{}                            % Fußzeile außen
\cfoot{\pagemark}                   % Fußzeile mitte
\ofoot{}                            % Fußzeile innen

%--------------------------
% Kopfzeile und Fußzeile für Seiten mit einem Kapitelanfang
%--------------------------
\newtriplepagestyle{chapterpagestyle}    % Neuen Style für Seiten mit Kapitelanfang definieren
[0em]                        % Dicke der äußeren Linien
[0.04em]                     % Dicke der inneren Linien
{\rightmark}                 % Kopfzeile außen
{}                           % Kopfzeile mitte
{}                           % Kopfzeile innen
{}                           % Fußzeile außen
{\pagemark}                  % Fußzeile mitte
{}                           % Fußzeile innen

\renewcommand{\chapterpagestyle}{chapterpagestyle}



%--------------------------
% Schusterjungen und Hurenkinder
%--------------------------
\clubpenalty = 10000             % schließt Schusterjungen aus
\widowpenalty = 10000            % schließt Hurenkinder aus (nur mit \displaywidowpenalty)
\displaywidowpenalty = 10000     % schließt Hurenkinder aus (nur mit \widowpenalty)

%--------------------------
% Quelltexte
%--------------------------
\lstset{ %
	backgroundcolor=\color{white},   % choose the background color; you must add \usepackage{color} or \usepackage{xcolor}; should come as last argument
	basicstyle=\footnotesize\ttfamily,        % the size of the fonts that are used for the code
	belowskip=-0.5\baselineskip,
	breakatwhitespace=false,         % sets if automatic breaks should only happen at whitespace
	breaklines=true,                 % sets automatic line breaking
	captionpos=b,                    % sets the caption-position to bottom
	commentstyle=\color{gray},    % comment style
	deletekeywords={...},            % if you want to delete keywords from the given language
	escapeinside={\%*}{*)},          % if you want to add LaTeX within your code
	extendedchars=true,              % lets you use non-ASCII characters; for 8-bits encodings only, does not work with UTF-8
	frame=single,	                   % adds a frame around the code
	keepspaces=true,                 % keeps spaces in text, useful for keeping indentation of code (possibly needs columns=flexible)
	keywordstyle=\color{blue},       % keyword style
	language=SQL,                    % the language of the code
	morekeywords={SHOW,DATABASE,DATABASES,SCHEMA,IF,LOCATION,COMMENT,WITH,DBPROPERTIES,TBLPROPERTIES,EXTENDED,CASCADE,USE,STRING,ARRAY,MAP,STRUCT,TABLES,RENAME,TO,PARTITION,COLUMNS,AFTER,REPLACE,ROW,FORMAT,DELIMITED,FIELDS,TERMINATED,BY,COLLECTION,ITEMS,TERMINATED,BY,MAP,KEYS,TERMINATED,BY,LINES,TERMINATED,BY,STORED,AS,TEXTFILE,SERDEPROPERTIES,PARTITIONED,INPATH,LOAD,DATA,OVERWRITE,DIRECTORY,create\_namespace,list\_namespace,alter\_namespace,describe\_namespace,drop\_namespace,create,list,describe,alter\_status,disable,alter,drop,put,delete,get,scan,DUPLICATE,UPSERT,CHANGE,BTREE,BIGINT,INFILE,ENCLOSED,SUM,AVG,exists,is\_disabled,is\_enabled},           % if you want to add more keywords to the set
	sensitive=true,
	numbers=left,                    % where to put the line-numbers; possible values are (none, left, right)
	numbersep=5pt,                   % how far the line-numbers are from the code
	numberstyle=\tiny,              % the style that is used for the line-numbers
	rulecolor=\color{black},         % if not set, the frame-color may be changed on line-breaks within not-black text (e.g. comments (green here))
	showspaces=false,                % show spaces everywhere adding particular underscores; it overrides 'showstringspaces'
	showstringspaces=false,          % underline spaces within strings only
	showtabs=false,                  % show tabs within strings adding particular underscores
	stepnumber=1,                    % the step between two line-numbers. If it's 1, each line will be numbered
	stringstyle=\color{red},     % string literal style
	tabsize=2,	                   % sets default tabsize to 2 spaces
	title=\lstname                   % show the filename of files included with \lstinputlisting; also try caption instead of title
}
\lstset{literate=
	{á}{{\'a}}1 {é}{{\'e}}1 {í}{{\'i}}1 {ó}{{\'o}}1 {ú}{{\'u}}1
{Á}{{\'A}}1 {É}{{\'E}}1 {Í}{{\'I}}1 {Ó}{{\'O}}1 {Ú}{{\'U}}1
{à}{{\`a}}1 {è}{{\`e}}1 {ì}{{\`i}}1 {ò}{{\`o}}1 {ù}{{\`u}}1
{À}{{\`A}}1 {È}{{\'E}}1 {Ì}{{\`I}}1 {Ò}{{\`O}}1 {Ù}{{\`U}}1
{ä}{{\"a}}1 {ë}{{\"e}}1 {ï}{{\"i}}1 {ö}{{\"o}}1 {ü}{{\"u}}1
{Ä}{{\"A}}1 {Ë}{{\"E}}1 {Ï}{{\"I}}1 {Ö}{{\"O}}1 {Ü}{{\"U}}1
{â}{{\^a}}1 {ê}{{\^e}}1 {î}{{\^i}}1 {ô}{{\^o}}1 {û}{{\^u}}1
{Â}{{\^A}}1 {Ê}{{\^E}}1 {Î}{{\^I}}1 {Ô}{{\^O}}1 {Û}{{\^U}}1
{œ}{{\oe}}1 {Œ}{{\OE}}1 {æ}{{\ae}}1 {Æ}{{\AE}}1 {ß}{{\ss}}1
{ű}{{\H{u}}}1 {Ű}{{\H{U}}}1 {ő}{{\H{o}}}1 {Ő}{{\H{O}}}1
{ç}{{\c c}}1 {Ç}{{\c C}}1 {ø}{{\o}}1 {å}{{\r a}}1 {Å}{{\r A}}1
{€}{{\euro}}1 {£}{{\pounds}}1
}


%==========================
% Einstellungen (durch Um-/Neudefinierung)
%==========================

\makeatletter
\g@addto@macro{\UrlBreaks}{\UrlOrds}
\makeatother



%--------------------------
% Abstand vor und nach der Kapitelüberschrift umdefinieren
%--------------------------
\renewcommand{\chapterheadstartvskip}{\vspace*{-0.5\baselineskip}}
\renewcommand{\chapterheadmidvskip}{\par\nobreak\vspace*{0.25\baselineskip}}
\renewcommand{\chapterheadendvskip}{\vspace*{1\baselineskip}}


%--------------------------
% Übersetzung für Abbildung und Tabelle wird geändert, damit in den Über-
% und Unterschriften alle Tabellen und Abbildungen nur noch die Akürzungen
% stehen
%--------------------------
\renewcaptionname{english}{\figurename}{Figure}
\renewcaptionname{english}{\tablename}{Table}
\renewcommand{\lstlistingname}{Code}
\renewcommand{\lstlistlistingname}{List of Codes}


%--------------------------
% Abstände der Tabellen und Bildern im lot und lof zwischen Kapiteln entfernt
%--------------------------
\makeatletter
\addtocontents{lof}{\protect\let\protect\addvspace\protect\@gobble}
\addtocontents{lot}{\protect\let\protect\addvspace\protect\@gobble}
\addtocontents{lol}{\protect\let\protect\addvspace\protect\@gobble}
\makeatother


%-------------------------
% Fix lstlistoflistings problems
%-------------------------
\makeatletter
\AtBeginDocument{%
	\let\c@lstlisting\relax
	\newlistof[chapter]{lstlisting}{lol}{\lstlistlistingname}
	\renewcommand{\cftlstlistingnumwidth}{3em}
}
\makeatother




%--------------------------
% Erweiterung des Befehls \frontmatter
% - Seitennummer
%--------------------------
\let\oldFrontmatter\frontmatter
\renewcommand*{\frontmatter}{%
	% Alte Definition
	\oldFrontmatter
	% Seitennummerierung mit großen römischen Zahlen
	\pagenumbering{Roman}
}

%--------------------------
% Tikz External Einstellungen
%--------------------------

\tikzexternalize[					% Unterordner vom Cache
	prefix=TikzCache/
]


\tikzexternaldisable
\newcommand{\inputtikz}[1]{
	%\tikzexternalenable
	%\tikzsetnextfilename{#1}
	\includegraphics{#1.pdf}
	%\tikzexternaldisable
}
